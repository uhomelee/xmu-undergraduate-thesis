% !TeX root = ../Main/XMU.tex
\chapter{总结}{Conclusion}
本文提出了一种利用深度学习神经网络求解IK问题的方法,特别是目前涉及此问题最为广泛的机械臂领域。此IK解算器能够处理涉及复杂人体-环境交互的场景,与此同时可以生成与真实动作最相似的骨骼姿态。本次毕业论文的主要贡献可以概括如下:
\begin{itemize}
  \item 将深度学习技术应用至该问题。不同于以往的其他论文中的工作,做到了大规模的数据集上的训练并得到了可靠的测试集结果。
  \item 定义了输入帧和输出帧之间的临界时间相关性,并将之与目前应用广泛的LSTM单元进行结合。
  \item 使用了恒定肢体比率的假设和相对坐标,消除了以往工作中的个体之间的差别性,可以真正做到无需给出肢体长度等先验信息的基础上进行IK问题的求解。
  \item 对IK问题的求解的正确性进行了定义,并对IK解算器两个组件的输出进行了降噪,且降噪结果明显。
  \item 对基于单目摄像机的3D人体骨骼工作进行了复现,并运用IK解算器和去噪滤波器对结果进行了改进。
\end{itemize}

对于未来的工作,可以分为以下X点:
\begin{itemize}
  \item 对四肢的IK解算器建立关联性(如利用树网络),做到全身的end-to-end的训练。
  \item 将其他信息(如图片语义信息)等加入网络训练中,以此来改善网络的求解自然度。
  \item 将本文的IK解算器应用于全身运动捕捉,并在末端执行器上安装惯性传感器,以此来改善合成动作的自然性。
  \item 继续其他网络的求解性,如利用强化学习来解决数据库规模问题等。
\end{itemize}
