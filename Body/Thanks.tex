% !TeX root = ../Main/XMU.tex


\chapter*{致谢}

论文的实验过程从18年初春开始,于19年初夏结束,前后一年有余。遗憾的是,本文完成时,其中的工作还未尽善尽美。正如同所有的科研工作一般,本文因为涉及研究性的命题,并非一帆风顺。在此期间,需要的是大量的时间进行知识的学习和应用。

因为,这是我在本科阶段接触的第一个研究性项目。郭诗辉老师作为我在实验室期间的导师,给予了我巨大的帮助,所以感谢郭诗辉老师在立题以来、甚至是第一次见面以来对我的引导、支持、耐心、鼓励与鞭策。项目初期,涉及研究性题目远远超过了我对本科生和自身能力的评估,是郭老师充分的信任和支持,包括了精神上和实验设备上的支持,让我有了继续进行的基础、信心和勇气。郭老师不仅在大方向上为我描绘了一项科研工作进行的流程,更在细节上给予了很多的帮助,是在疑惑时的解答,更是懈怠时的敦促。郭老师也远远超出我的毕业论文指导老师的定位,在跟随他的一年多时间内,让我对自己未来的方向、自己真正追求和热衷于的事业有了一个更为清楚的蓝图。郭老师的身体力行,也让我明白了做科研所需要的严谨、勤奋、不懈,而这些需要成为做科研的一种的习惯。其实我再多浓墨重彩的叙述,也道不尽初见以来郭老师对我的帮助。寥寥数语,或许词不达意,希望能表达出内心的想法。

若形容这篇论文的工作进程为一条路,它是坎坷蜿蜒的。郭老师是昼夜不停歇的光辉,而我的恋人,是一直陪伴的沿途风景,是黑夜里相依的手执灯盏,是日间追随的鸟儿,一直聆听,未断的陪伴。我仍然记得,很多个夜晚,是难以取得进展的日子里,跨过一千公里的电话带来的慰藉和温情。我的父母,像无怨的拐杖。大二暑假之后,因为课业和实验室的工作,回家愈来愈少,每每回到家中少了一分熟悉,多了一份亲切,我的父母没有任何的埋怨,感谢一直以来的对我的支持和谅解。还有实验室的师兄师弟师妹们带来的一路的欢声笑语,让人怀念。

每每写到致谢部分,总是难以下笔,并非无话可说。千言无语,一提笔时头绪皆无。今日谨记。谢谢可爱的你们!
% Get out of here. Go back to Rome.You're young...the world is yours. And I'm old.离开这里 去罗马 你还年轻 世界是你的 我已经老了
%
%
% 致谢语应以简短的文字对课题研究与论文撰写过程中曾直接给予帮助的人员(例如指导教师、答疑教师及其他人员)表示自己的谢意。
%
% 作为毕业论文提交时,应注意事项:致谢内容用小四号宋体。根据2016年2月施行的《厦门大学本科毕业论文(设计)规范》,致谢被放在论文起首。致谢结构一般分为三个部分:1,回顾;2,感谢; 3,承担责任以及献辞。第一部分可以简述论文写作的经历,所面对的挑战以及你如何应对。第二部分具体感谢在论文过程中给与你帮助的人。第三部分指出你将为自己的论文承担责任,如果你希望将此论文献给谁,可以在最后指出。致谢内容请亲自撰写,使其具备你个人的特色。抄袭任何模板内容是极其懒惰、没有意义、不负责任和错误的行为。
