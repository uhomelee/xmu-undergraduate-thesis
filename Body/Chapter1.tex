% !TeX root = ../Main/XMU.tex
\chapter{模板的使用说明}{The usage guide of this template}

\section{使用的前提}{Prediction}

为了使用该模板,需要安装一个TeX的发行版本。可以选择Texlive或者Miktex,他们都是跨平台的。而Texlive打包了比较多的宏包,较为庞大,Miktex则是临时下载没有的宏包。这里我推荐使用Miktex。但是对于Mac,推荐使用MacTeX,它是为Mac定制的发行版本,应该比较合适。特别提醒CTeX套装无法运行该模板。关于编译方式需选择XeLaTeX,否则无法正常编译该模板。


\section{几点说明}{Some notes}

为了正确使用该模板,请按照提示安装好可使用的TeX发行版本。因为论文内容比较多,因此采取了分文件的方式来构成该文档。主文档XMU.tex的位置位于Main下,正确编译后所得的pdf文件就在这里。Figure文件夹是存放图片的文件夹,该文件夹已经加入图片文件夹的位置,插入图片是无需多加路径,直接用文件名即可。关于Setting文件夹只需要把里面的Information.tex正确填入即可。而你需要编辑的仅有Body文件夹下的文件。

该模板是在厦门大学博士学位论文模板的基础上修改得到的,因为本科论文与博士学位论文的要求差别比较的,所以定制了该模板。由于本人水平有限,因此该模板写的并不好,但是应该勉强能够满足毕业论文的要求。但是仍然可能有许多错误的地方,希望各位使用者如果能发现错误之处能够提出。可以给我法邮件或者直接在github上面提issue。欢迎大家的参与,共同完善母校的模板。

由于本人是一名理科生,对文科的同学毕业论文的额外需求可能了解不多。虽说文科生用这个模板的可能性比较小,如果有人用,有额外的需求也可以提出。

联系方式:
邮箱: \href{mailto:camusecao@gmail.com}{camusecao@gmail.com}

github项目的地址 : \href{https://github.com/CamuseCao/XMU-Undergraduate-thesis-template}{XMU-Undergraduate-thesis-template}
