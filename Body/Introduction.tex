% !TeX root = ../Main/XMU.tex
\chapter{绪论}{Introduction}

\section{引言}{Introduction}
\textbf{正向运动学(forward kinematics)}指的是在给定特定角色姿势的3D空间中计算末端效应器位置的过程。\textbf{逆向运动学(inverse kinematics)}指的是当已知末端执行器放置在所需目标位置时,计算沿身体骨骼连接的关节方向的过程。反向运动学(IK)意在解决人形骨骼自由度难题,即已知骨骼末端的位置姿态,计算骨骼对应位置的全部关节变量。从计算机视觉和人工智能发展历史来看,在控制人形机器人和角色动画领域,逆向运动学始终是一个难以逾越的待解决问题。在机器人控制中使用该技术时最常见的场景就是机器人手臂指定末端位置抓取的待解决任务,除此之外,通过约束其其手和脚的位置,可以做到可视地展示虚拟角色以与环境交互过程。

然而,关于反向运动学的解决任务并不是一挥而就的。该问题最大的挑战在于空间冗余度\cite{reinhart2011neural}问题,即在末端执行器的位置相同的情况下,可能同时存在多个关于身体姿态的解。尽管所有的解在数值上都是正确的,但某些解对应的身体姿态看起来并不自然。这一问题严重影响了图像应用(如游戏和电影)中角色动画合成的自然性。在与真实人类交互的混合环境中的社交机器人方面,该问题也是日渐显著。

我们通过使用深度学习神经网络对姿势分布建模来解决IK问题。神经网络的训练集包含超过200万个采集的真人的姿势数据和以此为依据计算得到的末端执行器位置和关节方向之间的映射。现有的数据驱动方法通过利用隐藏模型中嵌入的先验信息解决IK问题,包括高斯过程隐变量模型(Gaussian Process Latent Variable Model,GPLVM)\cite{grochow2004Style},主成分分析(Principal Components Analysis,PCA)\cite{tournier2009motion},多变量高斯分布模型(Multi-variate Gaussian distribution model,MGDMs)\cite{huang2017multi}。但是,这些模型都只能处理中等大小以下规模的数据\cite{koker2004study,daya2010applying,feng2012inverse}。尽管目前已经有一些使用深度神经网络进行角色控制的工作\cite{levine2014learning,holden2016deep,holden2017phase},但我们并没有找到和我们工作一样的解决IK问题的应用开发。据我们所知,这项工作是首先使用相当大小的数据库训练IK问题解算器。更具体地说,本次毕业论文的贡献是:
\begin{itemize}
\item 我们优先考虑符合人体自然姿态的运动,并且更倾向于由跟踪的真人示范的自然姿态。据我们所知,本次毕业论文是最先利用这样一个大型数据库来解决IK问题。现有的方法深度学习技术来解决IK问题(特别是机器人手臂运动的IK问题)
\item 我们定义了输入动作帧和输出动作帧之间的临界时间相关性,并将短序动作序列作为网络输入来提高预测精度。除此之外,我们将一维的均值滤波器应用在输出关节通道以达到动作稳定的目的。
\item 我们使用了恒定肢体比率的假设和相对坐标。这是基于以下观察结果:股骨:胫骨的长度比和肱骨:尺骨的长度比非常相似(分别为1.21和1.22),个体之间的差异很小($\le7\%$)\cite{pietak2013fundamental}。这个结果说明可以消除肢体长度的影响,因此该工作可以适用于各种身体长度。
\end{itemize}

与传统的解析方案相比较,我们的方法能够避免预测出运动奇点和不可能的姿势。因为在传统方法中,病态矩阵的反演可能导致这些情况。 我们的解决方案做到了与解析方案相当的运行成本,并且比迭代求解器的表现更好。
